\documentclass[../main.tex]{subfiles}
\graphicspath{{\subfix{../images/}}}
\begin{document}


\subsubsection{Symetrie tenzoru napětí}

\begin{multline}
    \frac{d}{dt}\int_{\gamma(t)} (\vec{x} - \vec{x}_0 \varrho \vec{v} d\vec{x}) =\\
    \int_{\gamma(t)} \varrho (\vec{x}-\vec{x}_0) \times \vec{F} d\vec{x} + 
    \varepsilon_{ijk} \vec{e}_i \int_{\gamma(t)} (tau_{kj} + (x_j - x_{o,j}) \partial_l \tau_{kl}) d\vec{x}
\end{multline}

Rozeberme první člen pomocí RTT

\begin{multline}
    \frac{d}{dt}\int_{\gamma(t)} (\vec{x} - \vec{x}_0 \varrho \vec{v} d\vec{x}) =\\
    \int_{\gamma(t)} \varrho \frac{D}{Dt} ((\vec{x} - \vec{x}_0 \times \vec{V}))=\\
    \int_{\gamma(t)} \varrho \left( \frac{D (\vec{x} - \vec{x}_0)}{D_t}\times \vec{V} + (\vec{x} - \vec{x}_0) \times \frac{D\vec{V}}{Dt} \right) =\\
    \int_{\gamma(t)} \varrho \left( 0 + (\vec{x} - \vec{x}_0) \times \frac{D\vec{V}}{Dt} \right) =\\
    \int_{\gamma(t)} \varrho (\vec{x} - \vec{x}_0) \times \frac{D\vec{V}}{Dt} d\vec{x}
\end{multline}

Využili jsme
\begin{equation*}
    \frac{D (\vec{x} - \vec{x}_0)}{D_t}\times \vec{V} =     \frac{D \vec{x} }{D_t}\times \vec{V} =
    \vec{V} \times \vec{V} = \vec{0} 
\end{equation*}



Následně škálujeme $\gamma(t)$ s faktorem $\varepsilon$ ve všech rozměrech podle bodu $\vec{x}_0$
\begin{equation}
    \gamma(t)_\varepsilon = \{ \vec{x}_0 + \varepsilon(\vec{x} - \vec{x}_0) | \vec{x} \in \gamma(t) \}
\end{equation}

integrujeme přes $\gamma(t)_\varepsilon$ místo $\gamma(t)$ a $\varepsilon \rightarrow 0$, po vydělení zbývá jediný člen,
který nejde k 0 (protože nezávisí na $(\vec{x} - \vec{x}_0)$) a to po vydělení $\varepsilon^3$ je
\begin{equation}
    \frac{1}{\varepsilon^3} \varepsilon_{ijk} \vec{e}_i \int_{\gamma(t)_\varepsilon} \tau_{kj} d\vec{x}
\end{equation}
v limitě $\varepsilon \rightarrow 0$
\begin{equation}
    |\gamma(t)|\varepsilon_{ijk} \vec{e}_i \tau_{kj} (\vec{\xi})
\end{equation}

Platí 
$\varepsilon_{ijk} \vec{e}_i \tau_{kj} (\vec{x}_0) = 0$

Tento člen se rovná 0. Po složkách. 1. složka
\begin{equation}
    \varepsilon_{ijk}\tau_{jk} = 0
\end{equation}
\begin{equation}
    \tau_{23} = \tau_{32}
\end{equation}

Analogicky z 2. a 3. složky
\begin{equation}
    \tau_{13} = \tau_{31}, \tau_{12} = \tau{21}
\end{equation}

Z toho 
\begin{equation}
    \mathbb{T} = \mathbb{T}^T
\end{equation}

Interpretace je, že síly nezpůsobují roztáčení libovolně malého elementu $d\gamma$


\subsubsection{Rovnice bilance hybnosti v obecném tvaru}

\begin{equation}
    \frac{d}{dt} \int_{\gamma(t)} \varrho \vec{V} d\vec{x} = \int_{\gamma(t)} \varrho \vec{F} d\vec{x} + \int_{\partial \gamma (t)}  \vec{T}(t,\vec{x}, \vec{n}) dS 
\end{equation}

po složkách
\begin{equation}
    \frac{d}{dt} \int_{\gamma(t)} \varrho V_i d\vec{x} = \int_{\gamma(t)} \varrho F_i d\vec{x} + \int_{\partial \gamma (t)}  T_i(t,\vec{x}, \vec{n}) dS 
\end{equation}

Použijeme RTT \todo{dopsat co to je, na víc místech}

\begin{equation}
    int_{\gamma(t)} \varrho \frac{D V_i}{Dt} d\vec{x} = \int_{\gamma(t)} \varrho F_i d\vec{x} + \int_{\partial \gamma (t)}  \tau_{ij} n_j dS 
\end{equation}


Alternativně přes RTT
\begin{equation}
    \int_{\gamma(t)} \frac{\partial}{\partial t} (\varrho V_i) + \nabla \cdot (\varrho V_i \vec{V}) d\vec{x} = \int_{\gamma(t)} \partial_j \tau_{ij} + \varrho F_i d\vec{x}
\end{equation}

Protože $\gamma(t)$ je libovolně, rovnají se vnitřky integrálů

\begin{equation}
     \frac{\partial}{\partial t} (\varrho V_i) + \nabla \cdot (\varrho V_i \vec{V}) =  \partial_j \tau_{ij} + \varrho F_i 
\end{equation}

To je zákon zachování hybnosti v diferenciálním (konzervativním) tvaru, můžeme přejít zpátky k vektorům

\begin{equation}
    \frac{\partial(\varrho \vec{V})}{\partial t} + \nabla \cdot (\varrho \vec{V} \otimes \vec{V}) = \nabla \cdot \mathbb{T} + \varrho \vec{F}
\end{equation}


\subsection{Určení tvaru tenzoru tlaku}

\subsubsection{Jednoduché tekutiny}
\begin{equation}
    \mathbb{T} = - P \mathbb{I} + \mathbb{T}_D
\end{equation}

$P$ je hydrostatický (termodynamický) tlak, $\mathbb{T}_D$ je dynamický tenzor napětí (tenzor viskózního napětí)

Pokud    $\mathbb{T}_D = 0 \Leftrightarrow \vec{V} = 0 $ a derivace jsou $\vec{V}$ jsou 0.


Zákon zachování hybnosti pro jednoduché tekutiny
\begin{equation}
    \varrho \frac{D\vec{V}}{Dt} = -\nabla P + \nabla \cdot \mathbb{T}_D + \varrho \vec{F}
\end{equation}

Druhý člen jsme získali z
\begin{equation}
    \nabla\cdot(P\mathbb{I}) = (\partial_j (P \delta_{ij})) =  \partial_i P = \nabla P
\end{equation}

\subsubsection{Newtonovské tekutiny a Navierovy-Stokesovy rovnice}

$\mathbb{T}_D$ závisí na rychlostech $\vec{V}$ a jejích derivacích.

Definujme objektivní veličiny.

Ty by měly být stejné pro všechny pozorovatele, z toho máme $\vec{x}' = \vec{A}(t) + Q(t) \vec{x}$, $Q$ ortogonální.

Pak říkáme, že veličiny jsou objektivní právě pokud 
\begin{itemize}
    \item $\alpha(\vec{x}) = \alpha' (\vec{x}')$ pro skalár $\alpha$
    \item $||\vec{\alpha} (\vec{x})|| = ||\vec{\alpha}' (\vec{x}')||$ pro vektor $\vec{\alpha}$
\end{itemize}
Tenzorové pole $\mathbb{M}$ je objektivní právě když platí 

$(\mathbb{M} \vec{\alpha})' = \mathbb{M}' \vec{\alpha}'$, označme $\vec{\alpha} = \vec{y} - \vec{x}$  a$\vec{\alpha}' = \vec{y}' - \vec{x}'$ 

Pak 
$Q(\mathbb{M} \vec{\alpha}) = \mathbb{M}' Q \vec{\alpha}$ a $\mathbb{M} \vec{\alpha} = Q^T \mathbb{M}' Q \vec{\alpha}$.

Tedy $\vec{\alpha}' = Q \vec{\alpha}$

Newtonovská tekutina splňuje

\begin{itemize}
    \item Závislost $\mathbb{T}_D$ na složkách $\mathbb{D}$ je lineární
    \item Tekutina je izotropní, bez vnitřní struktury.
\end{itemize}

Druhý bod znamená
\begin{equation}
    \mathbb{T}_D'(\mathbb{D}) =Q^T \mathbb{T} (\mathbb{D}) Q = \mathbb{T}_D (\mathbb{D}') = \mathbb{T}_D (Q^T \mathbb{D} Q) 
\end{equation}, zřejmě $\mathbb{D}$ je izotropní funkce sebe sama a stejně tak $\mathbb{D}^n$:
\begin{equation}
    Q^T \mathbb{D}' Q = Q^T \mathbb{D} Q Q^T \mathbb{D} Q \dots = (Q^T \mathbb{D} Q)^n 
\end{equation}

Pokud uvažujeme závislost 
$\mathbb{T}_D (\mathbb{D}) = \sum_{n=0}^N \tilde{\alpha_n} \mathbb{D}^n$, 
kde $\tilde{\alpha}_n = \tilde{\alpha}_n (_\mathbb{D} \mathbb{I}_1, _\mathbb{D} \mathbb{I}_2, _\mathbb{D} \mathbb{I}_3)$, to jsou BÚNO hlavní invariaanty $\mathbb{D}$.


Protože platí Hamiltonova-Cayleyho věta, tj. 
\begin{equation}
    \mathbb{D}^3 + _\mathbb{D} \mathbb{I}_1 \mathbb{D} ^2 + _\mathbb{D} \mathbb{I}_2 \mathbb{D} + _\mathbb{D} \mathbb{I}_3 \mathbb{I} = 0 
\end{equation}

Tak lze $\mathbb{D}^3$ vypočítat pomocí $\mathbb{D}, \mathbb{D}^2$ lineární kombinací a indukcí nakonec i $\mathbb{D}^n$ lze vypočítat lineární kombinací předchozích členů.

Z toho plyne závislost $\mathbb{T}_D (\mathbb{D}) = \sum_{n=0}^N \tilde{\alpha}_n (_\mathbb{D} \mathbb{I}_1, _\mathbb{D} \mathbb{I}_2, _\mathbb{D} \mathbb{I}_3) \mathbb{D}^n$
je stejně  obecné jako $\mathbb{T}_D = \alpha_0 \mathbb{I} + \alpha_1 \mathbb{D} + \alpha_2 \mathbb{D}^2$, kde $\alpha_{1,2,3}$ jsou funkcemi invariantů.


Protože navíc chceme lineární závislost a 
\begin{itemize}
    \item $_\mathbb{D} \mathbb{I}_1 = Tr \mathbb{D}$
    \item $_\mathbb{D} \mathbb{I}_2 = \frac{1}{2} (Tr \mathbb{D})^2 - Tr(\mathbb{D}^2)$
    \item $_\mathbb{D} \mathbb{I}_3 = \det \mathbb{D}$
\end{itemize}

tak $\alpha_0$ může být lineární funkcí invariantu $_\mathbb{D} \mathbb{I}_1$, máme
\begin{itemize}
    \item $\alpha_0 = \mu ' Tr \mathbb{D}$
    \item $\alpha_1 = konst = 2\mu$
    \item $\alpha_2 = 0$
\end{itemize}

Z toho \begin{equation}
    \mathbb{T}_D = \mu' Tr \mathbb{D} \mathbb{I} + 2 \mu \mathbb{D}
\end{equation}
Z toho
\begin{equation}
    \mathbb{D} = \frac{1}{2}(\partial_i V_j + \partial_j V_i)
\end{equation}
\begin{equation}
    Tr \mathbb{D} = \partial_i V_i = \nabla\cdot \vec{V}
\end{equation}


A tedy \begin{equation}
    \mathbb{T}_D  = \left( \mu' \partial_k V_k \cdot \delta_{ij} + \mu (\partial_i V_j + \partial_j V_i)      \right)
\end{equation}

$\mu$ je koeficient dynamické viskozity, $\mu'$ je druhý viskózní koeficient



Dosazením tohoto tvaru $\mathbb{T}_D$ do zákona zachování hybnosti dostáváme Navierovy-Stokesovy rovnice pro stlačitelné a vazké proudění 
\begin{equation}
    \varrho\frac{D\vec{V}}{Dt} = - \nabla P + \nabla \cdot \mathbb{T}_D + \varrho \vec{F}
 \end{equation}


Speciální případy:
\begin{enumerate}
    \item nevazké proudění, $\mu,\mu' = 0$ dostaneme Eulerovy rovnice $\varrho \frac{D\vec{V}}{Dt} = -\nabla P + \varrho \vec{F}$ 
    \item Nestlačitelné proudění \begin{equation}
        |\gamma_0| = \int_{\gamma_0} 1 d\vec{X} != \int_{\gamma(t)} 1 d\vec{x} = \int_{\gamma_0} |det \mathbb{F}| d\vec{X} \implies |\det \mathbb{F} (t,\vec{X})|= 1 \text{ vždy}
    \end{equation} z toho \begin{equation}
        d 
    \end{equation}
\end{enumerate}








\end{document}